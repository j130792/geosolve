%%%%%%%%%%%%%%%%%%%%%%%%%%%%%%%%%%%%%%%%%%%%%%%%%%%%%%%%%%%%%%%%%%%%%%%
%% Hey Emacs,
%% mode:latex ***
%% tex-main-file: "invariant.tex"  ***
%%
%%
%%
%% James Jackaman (1)
%%
%%  (1) jjackaman@mun.ca
%%
%% Dear Human:
%%
%% This source file may be subject to copyright restrictions as
%% mandated by the Journal's policy and arXiv's copyright policies.
%%
%% In any case, you MUST SEEK PERMISSION from at least one of the
%% authors prior to any use of this source file.
%%
%%%%%%%%%%%%%%%%%%%%%%%%%%%%%%%%%%%%%%%%%%%%%%
\documentclass[final,reqno]{amsart}
\usepackage{pdfsync}
\usepackage{amsmath,amssymb}
\usepackage{enumerate}
\usepackage{xspace}
\usepackage{array}
\usepackage{tabularx}
\usepackage{booktabs}
%\usepackage{tabulary}
\usepackage{subfigure}
\usepackage{caption}
\usepackage{fullpage}
\usepackage{cite}
\usepackage{caption}
\usepackage{james}

\hypersetup{colorlinks, linkcolor=blue, citecolor=blue, urlcolor=blue}

% Commenting macros
\newcommand{\james}[1]{ {\color{magenta}[ J: #1 ]} }
\newcommand{\francis}[1]{ {\color{red}[ F: #1 ]} }
\newcommand{\alex}[1]{ {\color{green}[ A: #1 ]} }

%remove unneeded equation references
\usepackage{mathtools}
\mathtoolsset{showonlyrefs=true}


\setlength{\parindent}{12pt} %to have no indentation in the beginning
                            %of paragraph

\author{
  James Jackaman
}
\address{
  James Jackaman
  \thanks{
    Department of Mathematics and Statistics, Memorial University of
    Newfoundland, St.\ John's, NL, A1C 5S7, Canada
    {\tt{jjackaman@mun.ca}}.
  }}

\author{Scott MacLachlan}
\address{
  Scott MacLachlan
  \thanks{
    Department of Mathematics and Statistics, Memorial University of
    Newfoundland, St.\ John's, NL, A1C 5S7, Canada
    {\tt{smaclachlan@mun.ca}}.
  }}

\thanks{This research was supported by people who gave us money for stuff.}


%%%%%%%%%%%%%%%%%%%%%%%%%%%%%%%%%%%%%%%%%%%%%%%%%%%%%%%%%%%%%%%%%%%%%%%% 
\title[]%
{Geometrically constrained linear solvers}
\date{\today}
%%%%%%%%%%%%%%%%%%%%%%%%%%%%%%%%%%%%%%%%%%%%%%%%%%%%%%%%%%%%%%%%%%%%%%%%
%\pdfformat{true}
%%%%%%%%%%%%%%%%%%%%%%%%%%%%%%%%%%%%%%%%%%%%%%%%%%%%%%%%%%%%%%%%%%%%%%%%


\begin{document}

\begin{abstract}

  In this paper we do some stuff for some reasons.
 
\end{abstract}

\maketitle

%%%%%%%%%%%%%%%%%%%%%%%%%%%%%%%%%%%%%%%%%%%%%%%%%%%%%%%%%%%%%%%%%%%%%%%%

\section{Introduction} \label{sec:introduction}

This comes last.




\section{Methodology/ideas}



\section{Linear KdV}

The linear KdV equation is described by
\begin{equation}
  u_t + \frac12 u_x  + u_{xxx} = 0
  ,
\end{equation}
and has several non-trivial invariants. For the sake of exposition we
shall focus on two of these invariants, namely mass
\begin{equation}
  \ddt \int_\Omega u \di{x} = 0
\end{equation}
and energy
\begin{equation}
  \ddt \int_\Omega \frac12 u_x^2 - \frac12 u^2 \di{x} = 0
  .
\end{equation}


\subsection{Tabulation of results}

\begin{table}[h]
  \captionsetup{width=\linewidth}

  \tiny{
    \caption{$N=100$, $M=50$, $p=1$.} \begin{tabular}{rrrrrr}
\toprule
 gmres residual norm &  gmres mass deviation &  gmres energy deviation &  geosolve residual norm &  geosolve mass deviation &  geosolve energy deviation \\
\midrule
        1.556130e+01 &          5.235306e-01 &           -1.895241e-01 &            1.556130e+01 &             5.234831e-01 &              -1.894628e-01 \\
        7.768536e+00 &         -3.723425e-03 &            3.717766e-03 &            7.768714e+00 &             0.000000e+00 &              -5.473983e-06 \\
        1.793151e-01 &          4.294123e-05 &           -4.379511e-04 &            1.797840e-01 &            -7.105427e-15 &               0.000000e+00 \\
        1.632744e-01 &          3.914974e-05 &           -4.516552e-04 &            1.638355e-01 &            -7.105427e-15 &               0.000000e+00 \\
        9.846390e-04 &         -3.360754e-08 &           -1.866306e-05 &            1.159122e-03 &             1.421085e-14 &               0.000000e+00 \\
        1.570542e-08 &          1.421085e-14 &           -2.486900e-14 &            1.570544e-08 &             0.000000e+00 &               3.552714e-15 \\
        2.704677e-09 &         -2.842171e-14 &            3.552714e-14 &            2.704663e-09 &             7.105427e-15 &               0.000000e+00 \\
        5.589537e-10 &          2.131628e-14 &           -1.065814e-14 &            5.589543e-10 &             0.000000e+00 &               0.000000e+00 \\
        1.462773e-10 &         -4.973799e-14 &            6.750156e-14 &            1.462957e-10 &            -7.105427e-15 &               7.105427e-15 \\
        4.630150e-11 &          7.105427e-15 &            0.000000e+00 &            4.628434e-11 &            -7.105427e-15 &               0.000000e+00 \\
\bottomrule
\end{tabular}

    }
\end{table}

\begin{table}[h]
  \captionsetup{width=\linewidth}
  \tiny{
    \caption{$N=100$, $M=49$, $p=1$.} \begin{tabular}{rrrrrr}
\toprule
 gmres residual norm &  gmres mass deviation &  gmres energy deviation &  geosolve residual norm &  geosolve mass deviation &  geosolve energy deviation \\
\midrule
        1.619277e+01 &          5.437148e-01 &           -1.962652e-01 &            1.619277e+01 &             5.437241e-01 &              -1.962773e-01 \\
        7.845025e+00 &         -3.650905e-03 &            3.689034e-03 &            7.845199e+00 &             0.000000e+00 &               6.235793e-05 \\
        1.883427e-01 &          4.447909e-05 &           -4.367951e-04 &            1.887919e-01 &             0.000000e+00 &               3.552714e-15 \\
        1.715231e-01 &          4.070569e-05 &           -4.512847e-04 &            1.720629e-01 &             0.000000e+00 &               3.552714e-15 \\
        1.034298e-03 &         -3.349141e-08 &           -1.942740e-05 &            1.217904e-03 &             7.105427e-15 &               1.065814e-14 \\
        2.041052e-08 &         -2.842171e-14 &            2.131628e-14 &            2.041051e-08 &             0.000000e+00 &               7.105427e-15 \\
        3.055870e-09 &         -5.684342e-14 &            7.105427e-14 &            3.055884e-09 &             0.000000e+00 &               7.105427e-15 \\
        7.322206e-10 &         -2.131628e-14 &            1.421085e-14 &            7.322127e-10 &             0.000000e+00 &               0.000000e+00 \\
        1.673038e-10 &         -2.842171e-14 &            2.486900e-14 &            1.672861e-10 &            -7.105427e-15 &               1.421085e-14 \\
        4.577304e-11 &          2.131628e-14 &           -1.776357e-14 &            4.577186e-11 &            -7.105427e-15 &               1.065814e-14 \\
\bottomrule
\end{tabular}

    }
\end{table}

\begin{table}[h]
  \captionsetup{width=\linewidth}
  \tiny{
    \caption{$N=100$, $M=25$, $p=1$.} \begin{tabular}{rrrrrr}
\toprule
 gmres residual norm &  gmres mass deviation &  gmres energy deviation &  geosolve residual norm &  geosolve mass deviation &  geosolve energy deviation \\
\midrule
        7.128150e+01 &          1.693258e+00 &           -4.538772e-01 &            7.128150e+01 &             1.693516e+00 &              -4.542010e-01 \\
        1.074883e+01 &         -2.355575e-03 &            3.165565e-03 &            1.074894e+01 &             7.105427e-15 &               8.125343e-04 \\
        9.495860e-01 &          1.392533e-04 &           -3.397632e-04 &            9.496340e-01 &             0.000000e+00 &               7.105427e-15 \\
        8.810611e-01 &          1.374809e-04 &           -3.999854e-04 &            8.811466e-01 &             0.000000e+00 &              -3.552714e-15 \\
        5.056501e-03 &         -2.477715e-08 &           -6.921623e-05 &            5.960628e-03 &            -7.105427e-15 &               3.552714e-15 \\
        1.136915e-08 &          7.105427e-15 &           -3.552714e-15 &            1.136914e-08 &             0.000000e+00 &               3.552714e-15 \\
        2.601447e-09 &          2.131628e-14 &           -7.105427e-15 &            2.601483e-09 &             0.000000e+00 &               0.000000e+00 \\
        6.086951e-10 &         -7.105427e-15 &            1.776357e-14 &            6.086487e-10 &             7.105427e-15 &               3.552714e-15 \\
        1.618982e-10 &          2.131628e-14 &           -2.486900e-14 &            1.618871e-10 &             7.105427e-15 &               7.105427e-15 \\
        4.251749e-11 &          1.421085e-14 &           -1.421085e-14 &            4.249764e-11 &             0.000000e+00 &               3.552714e-15 \\
\bottomrule
\end{tabular}

    }
\end{table}

\begin{table}[h]
  \captionsetup{width=\linewidth}
  \tiny{
    \caption{$N=2$, $M=20$, $p=1$.} \begin{tabular}{rrrrrr}
\toprule
 gmres residual norm &  gmres mass deviation &  gmres energy deviation &  geosolve residual norm &  geosolve mass deviation &  geosolve energy deviation \\
\midrule
        9.122670e+00 &         -2.748044e-01 &            2.553585e+00 &            9.122671e+00 &            -2.663905e-01 &               2.543870e+00 \\
        3.631383e+00 &         -1.791576e+00 &            6.134082e-01 &            3.739596e+00 &            -7.105427e-15 &              -1.512542e+00 \\
        1.310285e+00 &          1.436390e-01 &           -5.764276e-01 &            1.524940e+00 &             0.000000e+00 &               1.065814e-14 \\
        7.536419e-01 &         -3.881288e-01 &           -3.225976e-01 &            1.442594e+00 &             0.000000e+00 &               0.000000e+00 \\
        2.086139e-01 &         -1.042247e-01 &            1.409948e-02 &            2.322089e-01 &            -2.842171e-14 &              -5.329071e-14 \\
        2.095289e-14 &         -7.105427e-15 &            3.552714e-15 &            1.080833e-14 &            -7.105427e-15 &               0.000000e+00 \\
        2.534497e-14 &         -4.973799e-14 &            5.684342e-14 &            8.676848e-15 &             0.000000e+00 &               0.000000e+00 \\
        2.985640e-14 &          7.105427e-15 &            1.065814e-14 &            8.851070e-15 &             0.000000e+00 &               3.552714e-15 \\
        2.733022e-14 &         -6.394885e-14 &            6.039613e-14 &            7.576459e-15 &             7.105427e-15 &               0.000000e+00 \\
        1.728477e-14 &         -4.263256e-14 &            3.907985e-14 &            8.361249e-15 &             0.000000e+00 &               3.552714e-15 \\
\bottomrule
\end{tabular}

    }
\end{table}

\begin{table}[h]
  \captionsetup{width=\linewidth}
  \tiny{
  \caption{$N=1000$, $M=500$, $p=1$.}
  \begin{tabular}{rrrrrr}
\toprule
 gmres residual norm &  gmres mass deviation &  gmres energy deviation &  geosolve residual norm &  geosolve mass deviation &  geosolve energy deviation \\
\midrule
        2.548347e+00 &          5.584898e-03 &           -2.179465e-03 &            2.548347e+00 &             5.583654e-03 &              -2.177845e-03 \\
        2.508511e+00 &         -2.068695e-04 &            1.054058e-04 &            2.508528e+00 &             7.105427e-15 &              -1.002693e-04 \\
        9.643026e-04 &          7.980091e-09 &           -4.644989e-06 &            1.079192e-03 &             7.105427e-15 &               2.486900e-14 \\
        9.347966e-04 &          4.081393e-09 &           -4.640725e-06 &            1.052893e-03 &             0.000000e+00 &               1.421085e-14 \\
        1.122533e-06 &         -2.394529e-12 &           -1.304084e-09 &            1.129059e-06 &            -7.105427e-15 &               4.263256e-14 \\
        4.219944e-07 &         -2.465583e-12 &           -1.296318e-09 &            4.388634e-07 &             0.000000e+00 &               4.973799e-14 \\
        3.842279e-07 &         -2.415845e-12 &           -1.293774e-09 &            4.026095e-07 &             0.000000e+00 &               2.486900e-14 \\
        3.806057e-07 &         -2.373213e-12 &           -1.291465e-09 &            3.990871e-07 &            -7.105427e-15 &               2.842171e-14 \\
        3.790126e-07 &         -2.359002e-12 &           -1.284114e-09 &            3.973547e-07 &             7.105427e-15 &               3.197442e-14 \\
        3.670988e-07 &         -2.245315e-12 &           -1.206871e-09 &            3.837559e-07 &             0.000000e+00 &               1.065814e-14 \\
        2.653935e-07 &         -1.179501e-12 &           -6.319034e-10 &            2.715113e-07 &            -7.105427e-15 &               8.881784e-14 \\
        9.972523e-08 &         -1.634248e-13 &           -8.939693e-11 &            1.000419e-07 &            -1.421085e-14 &               5.329071e-14 \\
        2.507688e-08 &         -1.421085e-14 &           -5.659473e-12 &            2.508206e-08 &             0.000000e+00 &               8.526513e-14 \\
        6.590073e-09 &          7.105427e-14 &           -5.080381e-13 &            6.590195e-09 &             0.000000e+00 &               4.618528e-14 \\
        1.805938e-09 &         -4.973799e-14 &            3.907985e-14 &            1.805940e-09 &            -7.105427e-15 &               2.486900e-14 \\
        4.673637e-10 &          7.105427e-14 &           -1.101341e-13 &            4.673511e-10 &            -1.421085e-14 &               2.131628e-14 \\
        1.374997e-10 &          2.842171e-14 &           -3.197442e-14 &            1.375667e-10 &            -7.105427e-15 &               2.131628e-14 \\
        5.903601e-11 &         -7.105427e-15 &           -7.105427e-15 &            5.915526e-11 &             7.105427e-15 &               1.065814e-14 \\
        4.861038e-11 &          3.552714e-14 &           -6.039613e-14 &            4.867245e-11 &             7.105427e-15 &               3.552714e-15 \\
        4.800655e-11 &         -7.105427e-15 &            0.000000e+00 &            4.818198e-11 &             0.000000e+00 &               1.776357e-14 \\
\bottomrule
\end{tabular}

  }
\end{table}


\section{Conclusion}

The introduction written in past tense.
  


\FloatBarrier
\bibliographystyle{abbrv}
\bibliography{geosolve}
\nocite{*}

\end{document}

% LocalWords:  tt th eqn prolongated sys ODEs Galerkin Bihlo NL AVF
% LocalWords:  invariantisation invariance equivariant isotropy ttt
% LocalWords:  foliation discretisation ty Valiquette Runge Kutta
% LocalWords:  equivariance movingframe invariantising bt groupoid
% LocalWords:  glode Painlev priori painleve painlevefe notational
% LocalWords:  fe vnl mlbf vectorial EOC linearised glodefe Sturm
% LocalWords:  iteratively invariantised discretisations symplectic
% LocalWords:  differentiable integrators meshless PDEs submanifolds
% LocalWords:  noproject invariantise projectable mfK submanifold
% LocalWords:  approximability Mathematica functionals ax sym
% LocalWords:  painlevemf normalisations lagrangebasis Liouville
% LocalWords:  ay eq Schwarzian Firedrake NumPy linalg linearisation
