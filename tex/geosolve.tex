%%%%%%%%%%%%%%%%%%%%%%%%%%%%%%%%%%%%%%%%%%%%%%%%%%%%%%%%%%%%%%%%%%%%%%%
%% Hey Emacs,
%% mode:latex ***
%% tex-main-file: "invariant.tex"  ***
%%
%%
%%
%% James Jackaman (1)
%%
%%  (1) jjackaman@mun.ca
%%
%% Dear Human:
%%
%% This source file may be subject to copyright restrictions as
%% mandated by the Journal's policy and arXiv's copyright policies.
%%
%% In any case, you MUST SEEK PERMISSION from at least one of the
%% authors prior to any use of this source file.
%%
%%%%%%%%%%%%%%%%%%%%%%%%%%%%%%%%%%%%%%%%%%%%%%
\documentclass[final,reqno]{amsart}
\usepackage{pdfsync}
\usepackage{amsmath,amssymb}
\usepackage{enumerate}
\usepackage{xspace}
\usepackage{array}
\usepackage{tabularx}
%\usepackage{tabulary}
\usepackage{subfigure}
\usepackage{caption}
\usepackage{fullpage}
\usepackage{cite}
\usepackage{caption}
\usepackage{james}

\hypersetup{colorlinks, linkcolor=blue, citecolor=blue, urlcolor=blue}

% Commenting macros
\newcommand{\james}[1]{ {\color{magenta}[ J: #1 ]} }
\newcommand{\francis}[1]{ {\color{red}[ F: #1 ]} }
\newcommand{\alex}[1]{ {\color{green}[ A: #1 ]} }

%remove unneeded equation references
\usepackage{mathtools}
\mathtoolsset{showonlyrefs=true}


\setlength{\parindent}{12pt} %to have no indentation in the beginning
                            %of paragraph

\author{
  James Jackaman
}
\address{
  James Jackaman
  \thanks{
    Department of Mathematics and Statistics, Memorial University of
    Newfoundland, St.\ John's, NL, A1C 5S7, Canada
    {\tt{jjackaman@mun.ca}}.
  }}

\author{Scott MacLachlan}
\address{
  Scott MacLachlan
  \thanks{
    Department of Mathematics and Statistics, Memorial University of
    Newfoundland, St.\ John's, NL, A1C 5S7, Canada
    {\tt{smaclachlan@mun.ca}}.
  }}

\thanks{This research was supported by people who gave us money for stuff.}


%%%%%%%%%%%%%%%%%%%%%%%%%%%%%%%%%%%%%%%%%%%%%%%%%%%%%%%%%%%%%%%%%%%%%%%% 
\title[]%
{Geometrically constrained linear solvers}
\date{\today}
%%%%%%%%%%%%%%%%%%%%%%%%%%%%%%%%%%%%%%%%%%%%%%%%%%%%%%%%%%%%%%%%%%%%%%%%
%\pdfformat{true}
%%%%%%%%%%%%%%%%%%%%%%%%%%%%%%%%%%%%%%%%%%%%%%%%%%%%%%%%%%%%%%%%%%%%%%%%


\begin{document}

\begin{abstract}

  In this paper we do some stuff for some reasons.
 
\end{abstract}

\maketitle

%%%%%%%%%%%%%%%%%%%%%%%%%%%%%%%%%%%%%%%%%%%%%%%%%%%%%%%%%%%%%%%%%%%%%%%%

\section{Introduction} \label{sec:introduction}

This comes last.




%   \begin{table}[h] \parbox{.45\linewidth}{%
% \captionsetup{width=\linewidth}
% \caption{The standard finite element approximation
%   \eqref{eqn:painlevefe} where \eqref{eqn:num:ex21} and
%   \eqref{eqn:num:ex22} hold with $T=10$. %We observe optimal %
%   convergence in each polynomial degree.
%   \label{tab:ex20}} \input{tables/example2invariant0.tex} }
% \hfill \parbox{.45\linewidth}{% \captionsetup{width=\linewidth}
%   \caption{The invariant finite element approximation
%     \eqref{eqn:painleve:invariant} where \eqref{eqn:num:ex21} and
%     \eqref{eqn:num:ex22} hold with $T=10$. %We observe optimal %
%     convergence in each polynomial degree, and that for all % simulations
%     the approximation is exact at the end time.
%     \label{tab:ex21}} \input{tables/example2invariant1.tex} }
% \end{table}


\section{Methodology/ideas}



\section{Linear KdV}

The linear KdV equation is described by
\begin{equation}
  u_t + \frac12 u_x  + u_{xxx} = 0
  ,
\end{equation}
and has several non-trivial invariants. For the sake of exposition we
shall focus on two of these invariants, namely mass
\begin{equation}
  \ddt \int_\Omega u \di{x} = 0
\end{equation}
and energy
\begin{equation}
  \ddt \int_\Omega \frac12 u_x^2 - \frac12 u^2 \di{x} = 0
  .
\end{equation}


\subsection{Tabulation of results}






\section{Conclusion}

The introduction written in past tense.
  


\FloatBarrier
\bibliographystyle{abbrv}
\bibliography{geosolve}
\nocite{*}

\end{document}

% LocalWords:  tt th eqn prolongated sys ODEs Galerkin Bihlo NL AVF
% LocalWords:  invariantisation invariance equivariant isotropy ttt
% LocalWords:  foliation discretisation ty Valiquette Runge Kutta
% LocalWords:  equivariance movingframe invariantising bt groupoid
% LocalWords:  glode Painlev priori painleve painlevefe notational
% LocalWords:  fe vnl mlbf vectorial EOC linearised glodefe Sturm
% LocalWords:  iteratively invariantised discretisations symplectic
% LocalWords:  differentiable integrators meshless PDEs submanifolds
% LocalWords:  noproject invariantise projectable mfK submanifold
% LocalWords:  approximability Mathematica functionals ax sym
% LocalWords:  painlevemf normalisations lagrangebasis Liouville
% LocalWords:  ay eq Schwarzian Firedrake NumPy linalg linearisation
